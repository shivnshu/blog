%!TEX TS-program = xelatex
%!TEX encoding = UTF-8 Unicode
% Awesome CV LaTeX Template for CV/Resume
%
% This template has been downloaded from:
% https://github.com/posquit0/Awesome-CV
%
% Author:
% Claud D. Park <posquit0.bj@gmail.com>
% http://www.posquit0.com
%
% Template license:
% CC BY-SA 4.0 (https://creativecommons.org/licenses/by-sa/4.0/)
%


%-------------------------------------------------------------------------------
% CONFIGURATIONS
%-------------------------------------------------------------------------------
% A4 paper size by default, use 'letterpaper' for US letter
\documentclass[11pt, a4paper]{cv}

% Configure page margins with geometry
\geometry{left=1.4cm, top=.8cm, right=1.4cm, bottom=1.8cm, footskip=.5cm}

% Specify the location of the included fonts
\fontdir[fonts/]

% Color for highlights
% Awesome Colors: awesome-emerald, awesome-skyblue, awesome-red, awesome-pink, awesome-orange
%                 awesome-nephritis, awesome-concrete, awesome-darknight
\colorlet{awesome}{awesome-concrete}
% Uncomment if you would like to specify your own color
% \definecolor{awesome}{HTML}{CA63A8}

% Colors for text
% Uncomment if you would like to specify your own color
% \definecolor{darktext}{HTML}{414141}
% \definecolor{text}{HTML}{333333}
% \definecolor{graytext}{HTML}{5D5D5D}
% \definecolor{lighttext}{HTML}{999999}

% Set false if you don't want to highlight section with awesome color
\setbool{acvSectionColorHighlight}{true}

% If you would like to change the social information separator from a pipe (|) to something else
\renewcommand{\acvHeaderSocialSep}{\quad\textbar\quad}


%-------------------------------------------------------------------------------
%	PERSONAL INFORMATION
%	Comment any of the lines below if they are not required
%-------------------------------------------------------------------------------
% Available options: circle|rectangle,edge/noedge,left/right
% \photo{./examples/profile.png}
\name{Shivanshu}{Singh}
\position{Final Year Undergraduate{\enskip\cdotp\enskip}}
\address{Indian Institution of Technology Kanpur}

\mobile{(+91) 8004610134}
\email{shivanshusinghraj@gmail.com}
\homepage{r4ndombits.me}
\github{shivnshu}
\linkedin{shivnshu}
% \gitlab{gitlab-id}
% \stackoverflow{SO-id}{SO-name}
% \twitter{@twit}
% \skype{skype-id}
% \reddit{reddit-id}
% \extrainfo{extra informations}

%\quote{``Be the change that you want to see in the world."}


%-------------------------------------------------------------------------------
\begin{document}

% Print the header with above personal informations
% Give optional argument to change alignment(C: center, L: left, R: right)
\makecvheader

% Print the footer with 3 arguments(<left>, <center>, <right>)
% Leave any of these blank if they are not needed
\makecvfooter
  {}
  {Shivanshu Singh~~~·~~~Curriculum Vitae}
  {\thepage}


\cvsection{Education}
\begin{cventries}

  \cventry
    {Bachelor of Technology, Major in Mechanical Engineering} % Degree
    {Indian Institution of Technology Kanpur} % Institution
    {Kanpur, India} % Location
        {2014-2018 (Expected)} % Date(s)
    {
      \begin{cvitems} % Description(s) bullet points
        \item {Cumulative Performance Index / CGPA: 8.1/10}
      \end{cvitems}
    }

\end{cventries}

\cvsection{Work Experience}


\begin{cventries}

%---------------------------------------------------------
  \cventry
    {Software Intern, AutoDesk (Received Pre-Placement Offer)} % Job title
    {Auto Scaling of Docker containers} % Organization
    {Pune, India} % Location
    {May. 2017 - July. 2017} % Date(s)
    {
      \begin{cvitems} % Description(s) of tasks/responsibilities
        \item {Worked on adding the auto-scaling functionality for containers of DC/OS cluster running microservices}
        \item {Designed, implemented and deployed the required architecture in AWS cloud. (using Lambda functions, Redis database, CloudWatch and a SNS topic)}
        \item {Employed a machine learning technique to make the calculated guess of amount of scaling (up/down) for each service}
      \end{cvitems}
    }

%---------------------------------------------------------
\end{cventries}

\cvsection{Projects}

\begin{cventries}

%---------------------------------------------------------
  \cventry
    {Course Project, Prof. Debadatta Mishra} % Affiliation/role
    {Secure Memory Deduplication and Covert Channel Construction in Linux Kernel} % Organization/group
    {IIT Kanpur} % Location
    {Sep. 2017 - Nov. 2017} % Date(s)
    {
      \begin{cvitems} % Description(s) of experience/contributions/knowledge
        \item {}
      \end{cvitems}
    }

%---------------------------------------------------------
  \cventry
    {Course Project, Prof. Vinay Namboodiri} % Affiliation/role
    {Large Margin Multi-Modal RGBD Object Recognition} % Organization/group
    {IIT Kanpur} % Location
    {Sep. 2017 - Nov. 2017} % Date(s)
    {
      \begin{cvitems} % Description(s) of experience/contributions/knowledge
        \item {}
      \end{cvitems}
    }

%---------------------------------------------------------
  \cventry
    {Course Project, Prof. Sandeep Shukla} % Affiliation/role
    {Securing Zoobar Server} % Organization/group
    {IIT Kanpur} % Location
    {Jan. 2017 - Apr. 2017} % Date(s)
    {
      \begin{cvitems} % Description(s) of experience/contributions/knowledge
        \item {Simulated various exploitations in a web application called Zoobar, written in C and serving CGI scripts}
        \item {Employed control hijacking techniques like buffer overflow, integer overflow and format string attacks to exploit the vulnerabilities}
        \item {Performed various browser-based attacks like SQL injection, XSS, CSRF on Zoobar web application}
        \item {Fixed security bugs in web server, implemented privilege separation and server-side sand-boxing}
        \item {Demonstrated the limitations of various mitigation techniques like stack canaries, address space layout randomization (ASLR) etc.}
      \end{cvitems}
    }

% --------------------------------------------------------
\end{cventries}

\cvsection{Hackathons}

\begin{cventries}

%---------------------------------------------------------
  \cventry
    {Winner, 24-hours hackathon, Microsoft's Code.Fun.Do} % Affiliation/role
    {AMR (Advanced Recognition System)} % Organization/group
    {} % Location
    {Apr. 2017} % Date(s)
    {
      \begin{cvitems} % Description(s) of experience/contributions/knowledge
        \item {}
      \end{cvitems}
    }

%
%---------------------------------------------------------
  \cventry
    {Winner, 24-hours hackathon, Google Developer Group} % Affiliation/role
    {HashTag} % Organization/group
    {} % Location
    {Nov. 2016} % Date(s)
    {
      \begin{cvitems} % Description(s) of experience/contributions/knowledge
        \item {}
      \end{cvitems}
    }

% --------------------------------------------------------
\end{cventries}

\cvsection{Mini Projects}

\begin{cventries}

%---------------------------------------------------------
    \begin{cvitems} % Description(s) of experience/contributions/knowledge
        \item {Implemented various locking mechanisms like spin lock, semaphores, sequencial lock, RCU etc. in linux kernel and compared them empirically}
        \item {Built chat room for processes, when in kernel mode, using a char device and employing the monolithic nature of linux kernel}
        \item {Studied the various proof methods like coinduction, fusion etc. for corecursive programs of functional languages and gave a presentation}
        \item {Conceptualised and designed a responsive website for Cultural Council using CodeIgniter MVC framework, jQuery etc.}
    \end{cvitems}

%---------------------------------------------------------
\end{cventries}


\cvsection{Technical Skills}


\begin{cvskills}

%---------------------------------------------------------
  \cvskill
    {Programming} % Category
    {C/C++, Python, Haskell, bash scripting} % Skills

%---------------------------------------------------------
  \cvskill
    {Software and utilities} % Category
    {Linux shell utilities, Docker, Vim, Emacs, \LaTeX, PyTorch, ROS} % Skills

%---------------------------------------------------------
  \cvskill
    {Operating Systems} % Category
    {Arch Linux (with i3wm and xmonad), Debian Linux (Ubuntu, Kali)} % Skills

%---------------------------------------------------------
\end{cvskills}



%\input{cv/education.tex}
%\input{cv/skills.tex}
%\input{cv/experience.tex}
%\input{cv/extracurricular.tex}
%\input{cv/honors.tex}
%\input{cv/presentation.tex}
%\input{cv/writing.tex}
%\input{cv/committees.tex}


%-------------------------------------------------------------------------------
\end{document}
