%!TEX TS-program = xelatex
%!TEX encoding = UTF-8 Unicode
% Awesome CV LaTeX Template for CV/Resume
%
% This template has been downloaded from:
% https://github.com/posquit0/Awesome-CV
%
% Author:
% Claud D. Park <posquit0.bj@gmail.com>
% http://www.posquit0.com
%
% Template license:
% CC BY-SA 4.0 (https://creativecommons.org/licenses/by-sa/4.0/)
%


%-------------------------------------------------------------------------------
% CONFIGURATIONS
%-------------------------------------------------------------------------------
% A4 paper size by default, use 'letterpaper' for US letter
\documentclass[11pt, a4paper]{cv}

% Configure page margins with geometry
\geometry{left=1.4cm, top=.8cm, right=1.4cm, bottom=1.8cm, footskip=.5cm}

% Specify the location of the included fonts
\fontdir[fonts/]

% Color for highlights
% Awesome Colors: awesome-emerald, awesome-skyblue, awesome-red, awesome-pink, awesome-orange
%                 awesome-nephritis, awesome-concrete, awesome-darknight
\colorlet{awesome}{awesome-darknight}
% Uncomment if you would like to specify your own color
% \definecolor{awesome}{HTML}{CA63A8}

% Colors for text
% Uncomment if you would like to specify your own color
% \definecolor{darktext}{HTML}{414141}
% \definecolor{text}{HTML}{333333}
% \definecolor{graytext}{HTML}{5D5D5D}
% \definecolor{lighttext}{HTML}{999999}

% Set false if you don't want to highlight section with awesome color
\setbool{acvSectionColorHighlight}{true}

% If you would like to change the social information separator from a pipe (|) to something else
\renewcommand{\acvHeaderSocialSep}{\quad\textbar\quad}


%-------------------------------------------------------------------------------
%	PERSONAL INFORMATION
%	Comment any of the lines below if they are not required
%-------------------------------------------------------------------------------
% Available options: circle|rectangle,edge/noedge,left/right
% \photo{./examples/profile.png}
\name{Shivanshu}{Singh}
\position{Final Year Undergraduate}%{\enskip\cdotp\enskip}}
\address{Indian Institute of Technology Kanpur}

\mobile{(+91) 8004610134}
\email{shivansh@iitk.ac.in}
\homepage{randombits.xyz}
\github{shivnshu}
\linkedin{shivnshu}
% \gitlab{gitlab-id}
% \stackoverflow{SO-id}{SO-name}
% \twitter{@twit}
% \skype{skype-id}
% \reddit{reddit-id}
% \extrainfo{extra informations}

%\quote{``Be the change that you want to see in the world."}


%-------------------------------------------------------------------------------
\begin{document}

% Print the header with above personal informations
% Give optional argument to change alignment(C: center, L: left, R: right)
\makecvheader

% Print the footer with 3 arguments(<left>, <center>, <right>)
% Leave any of these blank if they are not needed
\makecvfooter
  {}
  {Shivanshu Singh~~~·~~~Curriculum Vitae}
  {\thepage}


\cvsection{Education}
\begin{cventries}

  \cventry
    {Bachelor of Technology, Major in Mechanical Engineering} % Degree
    {Indian Institute of Technology Kanpur} % Institution
    {Kanpur, India} % Location
        {2014-2018 (Expected)} % Date(s)
    {
      \begin{cvitems} % Description(s) bullet points
        \item {Cumulative Performance Index / CGPA: 8.1/10}
      \end{cvitems}
    }

\end{cventries}

\cvsection{Work Experience}


\begin{cventries}

%---------------------------------------------------------
  \cventry
    {Software Intern, AutoDesk (Received Pre-Placement Offer)} % Job title
    {Auto Scaling of Microservices running on Docker Containers} % Organization
    {Pune, India} % Location
    {May. 2017 - July. 2017} % Date(s)
    {
      \begin{cvitems} % Description(s) of tasks/responsibilities
        \item {Worked on adding the auto-scaling functionality for containers of DC/OS cluster running microservices.}
        \item {Built a modular design with two components, one responsible for fetching and arranging metrices and, other for decision making and scaling.}
        \item {Used separate AWS lambda functions for components and AWS SNS service for communication between them.}
        \item {Maintained designed state variables of micro-services in Redis database and used AWS CloudWatch for periodic invocation of first component.}
        \item {Employed a machine learning technique (Stochastic Gradient Descent), in the second component, to make the calculated guess of amount of scaling (up/down) for each micro-service.}
      \end{cvitems}
    }

%---------------------------------------------------------
\end{cventries}

\cvsection{Projects}

\begin{cventries}

%---------------------------------------------------------
  \cventry
    {Course Project, Prof. Debadatta Mishra} % Affiliation/role
        {\href{http://home.iitk.ac.in/~shivansh/resources/secure-deduplication.pdf}{Secure Memory Deduplication and Covert Channel Construction in Linux Kernel}} % Organization/group
    {IIT Kanpur} % Location
    {Sep. 2017 - Nov. 2017} % Date(s)
    {
      \begin{cvitems} % Description(s) of experience/contributions/knowledge
        \item {Studied the source code of KSM, a linux kernel(v4.14.4) thread responsible for implementing memory deduplication in linux operating systems.}
        \item {Constructed a covert channel for communication between two processes running on same physical machine, separated by VMs.}
        \item {Exploited the write-time differences between merged and unmerged pages, caused by COW(copy-on-write) on merged pages, for covert channel construction.}
        \item {Designed and developed a synchronized protocol for reliable communication between processes using two mutually known pages content.}
        \item {Simulated the information disclosure attack, caused by merging, to detect the content of javascript object typed array in web browser.}
        \item {Devised and developed two mitigation techniques, \textit{stochastic false merging} and \textit{stochastic false unmerging}, to introduce significant amount of noise in write time, thus making the covert channel unreliable.}
      \end{cvitems}
    }

%---------------------------------------------------------
  \cventry
    {Course Project, Prof. Vinay Namboodiri} % Affiliation/role
        {\href{http://home.iitk.ac.in/~shivansh/resources/multimodal-rgbd.pdf}{Large Margin Multi-Modal RGBD Object Recognition}} % Organization/group
    {IIT Kanpur} % Location
    {Sep. 2017 - Nov. 2017} % Date(s)
    {
      \begin{cvitems} % Description(s) of experience/contributions/knowledge
      \item {Worked on improving the sematic information in feature vector representation and consequently classification accuracy of images by leveraging the additional depth channel along with color channels based on \href{http://ieeexplore.ieee.org/iel7/6046/7302100/07258382.pdf}{this} paper.}
        \item {Developed a generic convolutional neural network (CNN) that takes two modals as input, processes them separately, correlate them using two correlation matrices and output their linear combination.}
        \item {Used a cost/loss function consisting of weighted loss in individual modal CNNs and designed loss incurred while calculating the correlational matrices.}
        \item {Employed an alternating approach for optimizing the weights assigned to individual modal in loss function and the correlational matrices.}
        \item {Demonstrated successfully the increased classification accuracy when compared to accuracies obtained by using only one modality or several modality but not correlating them.}
      \end{cvitems}
    }

%---------------------------------------------------------
  \cventry
    {Course Project, Prof. Sandeep Shukla} % Affiliation/role
    {Securing Zoobar Server} % Organization/group
    {IIT Kanpur} % Location
    {Jan. 2017 - Apr. 2017} % Date(s)
    {
      \begin{cvitems} % Description(s) of experience/contributions/knowledge
        \item {Simulated various exploitations in a web application called Zoobar, written in C and serving CGI scripts.}
        \item {Employed control hijacking techniques like buffer overflow, integer overflow and format string attacks to exploit the vulnerabilities.}
        \item {Performed various browser-based attacks like SQL injection, XSS, CSRF on Zoobar web application.}
        \item {Fixed security bugs in web server, implemented privilege separation and server-side sand-boxing.}
        \item {Demonstrated the limitations of various mitigation techniques like stack canaries, address space layout randomization (ASLR) etc.}
      \end{cvitems}
    }

% --------------------------------------------------------
\end{cventries}

\cvsection{Research Interests}

\begin{cventries}
    \begin{flushleft}
        Broadly Interested in Computer Systems, Cyber-Security and Networking
    \end{flushleft}
\end{cventries}

\cvsection{Hackathons}

\begin{cventries}

%---------------------------------------------------------
  \cventry
  {\textbf{Winner}, 24-hours hackathon, Microsoft's Code.Fun.Do} % Affiliation/role
        {\href{https://github.com/shivnshu/AMR-System}{AMR (Advanced Motion Recognition) System}} % Organization/group
    {} % Location
    {Apr. 2017} % Date(s)
    {
      \begin{cvitems} % Description(s) of experience/contributions/knowledge
        \item {Designed and developed an android app to make the android device a 3D controller to control a uploaded CAD model.}
        \item {Used three.js to load and render the model on the browser exposing the APIs to control its orientation and location on the display.}
        \item {Established a real time and reliable communication between the android device and the web browser using websockets.}
      \end{cvitems}
    }

%
%---------------------------------------------------------
  \cventry
  {\textbf{Winner}, 24-hours hackathon, Google Developer Group} % Affiliation/role
        {\href{https://github.com/shivnshu/hashtag}{HashTag}} % Organization/group
    {} % Location
    {Nov. 2016} % Date(s)
    {
      \begin{cvitems} % Description(s) of experience/contributions/knowledge
        \item {Implemented a file management system on the android device using tags that has superior accessibility capabilities as opposed to conventional file management system.}
        \item {Used linked list semantic to manage the mapping between files/folders and tags, and stored them into the SQLite database.}
      \end{cvitems}
    }

% --------------------------------------------------------
\end{cventries}

\cvsection{Mini Projects}

\begin{cventries}

%---------------------------------------------------------
    \begin{cvitems} % Description(s) of experience/contributions/knowledge
        \item {Implemented various locking mechanisms like spin lock, semaphores, sequencial lock, RCU etc. in linux kernel and compared their efficiency empirically.}
        \item {Built \href{https://github.com/shivnshu/chatroom}{chat room} for processes, when in kernel mode, using a char device and employing the monolithic nature of linux kernel.}
        \item {Studied and presented various proof methods like coinduction, fusion etc. for corecursive programs of functional languages. \href{http://home.iitk.ac.in/~shivansh/resources/proofmethods-corecursion.pdf}{ppt.}}
        \item {Conceptualised and designed a responsive website for Cultural Council, IIT Kanpur using CodeIgniter MVC framework, jQuery etc.}
    \end{cvitems}

%---------------------------------------------------------
\end{cventries}


\cvsection{Technical Skills}


\begin{cvskills}

%---------------------------------------------------------
  \cvskill
    {Programming} % Category
    {C, Java, Python, Haskell, bash scripting} % Skills

%---------------------------------------------------------
  \cvskill
    {Software and utilities} % Category
    {Linux shell utilities, Docker, Vim, Emacs, \LaTeX, PyTorch, ROS} % Skills

%---------------------------------------------------------
  \cvskill
    {Operating Systems} % Category
    {Arch Linux (with i3wm and xmonad), Debian Linux (Ubuntu, Kali)} % Skills

%---------------------------------------------------------
\end{cvskills}

\cvsection{Relevant Courses}
\begin{cventries}
        \begin{tabular}{l l l}
                Computer Systems Security (A)& Computer Organization* & Introduction to Programming (A)\\
                Linux Kernel Programming (A) & Principles of Programming Languages (A) & Modern Cryptography*\\
                \textit{\small {\color{lightgray}A: Top Grade, * ongoing courses}}
        \end{tabular}
\vspace{-0.3cm}
\end{cventries}

\cvsection{Miscellaneous}

\begin{cventries}
        \begin{cvitems}
        \item {Write blog about linux, books and programming in general at \href{http://randombits.xyz}{randombits.xyz}.}
        \item {Volunteered in CSAW (Cyber Security Awareness Weekend) organised by NYU and IIT Kanpur.}
        \end{cvitems}
\end{cventries}

%-------------------------------------------------------------------------------
\end{document}
